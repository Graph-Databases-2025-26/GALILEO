\documentclass[12pt,a4paper]{article}

% Useful packages
\usepackage[utf8]{inputenc}
\usepackage[italian]{babel}
\usepackage[T1]{fontenc}
\usepackage{graphicx}
\usepackage{amsmath}
\usepackage{hyperref}
\usepackage{booktabs}
\usepackage{array}

% Margini
\usepackage[margin=2.5cm]{geometry}

% Document Informations
\title{Report for deliverable with deadline 31 october 2025}
\author{Francesco Pivotto 2158296\\ Giorgia Amato 2159999\\ Alessio Demo 2142885 }
\date{\today}

\begin{document}

\maketitle

\tableofcontents
\newpage

\section{Statistics for the datasets in the experiments}

Table~\ref{tab:datasets} provides an overview of the datasets used our project. The datasets are divided into two main categories based on the type of task: IK (Incomplete Knowledge) and MC (Multiple Choice).
The MC-type datasets include PREMIER, sourced from BBC, and FORTUNE, based on Kaggle data.
Finally, GEO-TEST represents the test dataset, also derived from Spider [68], used to validate the model's performance.

\begin{table}[h]
\centering
\begin{tabular}{lcccc}
\toprule
\textbf{Dataset} & \textbf{Dataset} & \textbf{\# of} & \textbf{Avg. expected} & \textbf{Type} \\
\textbf{name} & \textbf{source} & \textbf{queries} & \textbf{cells} & \\
\midrule
FLIGHT2 & Spider [68] & 3 & 267.5 & IK \\
FLIGHT4 & Spider [68] & 3 & 267.5 & IK \\
FORTUNE & Kaggle & 10 & 7.9 & MC \\
GEO & Spider [68] & 32 & 22.8 & IK \\
MOVIES & IMDB & 9 & 54.7 & IK \\
PREMIER & BBC & 5 & 57.8 & MC \\
PRESIDENTS & Wiki & 26 & 42.2 & IK \\
WORLD & Spider [68] & 4 & 33.2 & IK \\



\midrule
GEO-TEST & Spider [68] & 10 & 24.1 & IK \\
\bottomrule
\end{tabular}
\caption{DDescription of used datasets}
\label{tab:datasets}
\end{table}

\section{Corpo del documento}
Contenuto principale.

\subsection{Sottosezione}
Dettagli aggiuntivi.

\section{Conclusioni}
Considerazioni finali.

\begin{thebibliography}{9}
\bibitem{esempio}
Autore, \emph{Titolo}, Editore, Anno.
\end{thebibliography}

\end{document}